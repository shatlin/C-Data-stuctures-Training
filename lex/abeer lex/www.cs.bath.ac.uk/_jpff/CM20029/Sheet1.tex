\documentstyle[fullpage,11pt]{article}
\begin{document}
\title{ \large \bf MATH20029: Example Sheet 1}
\author{John ff}
%%\date{}
\maketitle
\setlength{\parskip}{0.2in}

\vskip -1in

\begin{enumerate}
\item  Using LEX and possibly associated C, write a lexical analyser
  for the following subset of a C-like language:
  \begin{itemize}
    \item Comments starting with a {\tt //} and lasting to the end of the
      line, 
    \item Variables made from upper, lower-case letters and digits only,
      but starting with a letter
    \item Integers in decimal notation with an optional leading sign
    \item Strings, enclosed in {\tt "} characters, but no escaped
      characters ({\em i.e.}~no {\tt $\backslash$n})
    \item The keywords {\tt if else while int void return}
    \item The operators and punctuation {\tt ( ) , ; = * / || \&\& }
  \end{itemize}
  Ensure that when your lexer has recognised a token it prints the
  lexeme.

  \item Write a function that takes a string and returns a positive
  integer that can be used as a hash code.  Test it on some text files
  to show what the distribution of hash values is; do the same mod
  1024

  \item Write a hash table that can hold unique copies of strings, and
    provides the two functions
    \begin{itemize}
    \item void add\_string(char *);
    \item char *lookup(char *);
    \end{itemize}
    The second function should return the string stored in the hash
    system or NULL if it is not present.

    \item Modify your LEX program to add the variables to the hash
    table you have written

    \item Modify your symbol table to use a token structure rather
    than strings.

    \item Modify your LEX program to create a token-type structure
    rather than just a recognition.

\end{enumerate}


\end{document}
